\chapter{Vorwort}

Für Laien in buddhistischen Ländern ist es seit langem Tradition an
jedem Neu- und Vollmond ein Kloster am Ort zu besuchen, um sich einen
Dhamma-Vortrag anzuhören. Sogar der Buddha selbst ermutigte seine Sangha
diese vierzehntägige Praxis zu pflegen. Als mir vorgeschlagen wurde, das
Internet zu nutzen, um zu jedem Neu- und Vollmond kurze
Dhamma-Reflexionen zu verschicken, war ich von dieser Idee zunächst
nicht überzeugt, beschloss dann aber doch es auszuprobieren. Obwohl wir
in einer Welt leben, in der die Mondphasen keine große Bedeutung mehr
haben, helfen solche Reflexionen auch heute noch vielen Menschen, an die
alte Tradition, der wir angehören, erinnert zu werden.

Im September 2007 begannen wir Verse aus dem Dhammapada zu verschicken,
die wir aus \emph{A Dhammapada for Contemplation} (2006) ausgewählt hatten. An
jedem `Mond-Tag' wurde ein Vers angeboten, der durch einen kurzen
Kommentar erläutert wurde. Dieses Programm ist mittlerweile durch
Mundpropaganda und weitergeleitete E-Mails recht bekannt geworden. Ich
höre von Menschen aus verschiedenen Teilen der Welt, dass sie es
schätzen, von Zeit zu Zeit eine Erinnerung an eine alte Lebensweise zu
erhalten, während sie ihrem geschäftigen Leben nachgehen. Andere freuen
sich an jedem Neu- und Vollmond darauf, ihre E-Mails zu öffnen, wenn sie
abends von der Arbeit heimkommen. Diese Dhamma-Reflexionen werden privat
genutzt und vielfach kopiert, übersetzt und herumgereicht. Ich habe auch
gehört, dass sie des Öfteren als Diskussionsgrundlage für
Meditationsgruppen dienen. Indem ich meine persönlichen Reflexionen auf
diesem Wege mit Anderen teile, möchte ich sie ermutigen, ihre eigenen
kontemplativen Fähigkeiten zu nutzen. Es scheint unter praktizierenden
Buddhisten im Westen die Tendenz zu geben, Frieden und Verständnis
finden zu wollen, indem sie versuchen ihre Gedanken anzuhalten. Doch der
Buddha spricht davon, dass wir die wahre Natur unseres Geistes durch
`yoniso manasikara' oder weises Betrachten erkennen werden, nicht allein
dadurch, dass wir aufhören zu denken.

Ich möchte mich bei allen bedanken, die bei der Vorbereitung dieses
Materials geholfen haben. Für die Dhammapada-Verse habe ich verschiedene
verlässliche Übersetzungen herangezogen. Insbesondere habe ich die
Arbeit von Ven. Narada Thera (BMS 1978), Ven. Ananda Maitreya Thera
(Lotsawa 1988), Daw Tin Mya und den Herausgebern der burmesischen
Pitaka-Vereinigung (1987) und von Ajahn Thanissaro verwendet. Für die
(kommentariell) überlieferten Geschichten zu den Versen habe ich auch
die Internet-Seite www.tipitaka.net als Quelle genutzt.

Als ich von zahlreichen Lesern gehört hatte, dass eine Buch-Version
dieser Reflexionen nützlich wäre, habe ich mich an meinen guten Freund
Ron Lumsden gewandt. Sein beachtliches Geschick für redaktionelle
Bearbeitung hat dazu beigetragen, meine Arbeit einem größeren Publikum
zugänglich zu machen.

Möge das Gute, das aus der Zusammenstellung dieses kleinen Büchleins
entsteht, mit allen, die an der Produktion und dem Sponsoring beteiligt
waren, geteilt werden. Mögen alle, die den Weg suchen, ihn finden und
die Freiheit am Ende des Weges erleben. Mögen alle Wesen den Weg suchen.

\bigskip

{\par\raggedleft
Bhikkhu Munindo\\
Aruna Ratanagiri, Regenzeitretreat 2009
\par}

