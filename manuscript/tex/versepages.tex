%% == 1 ==

\begin{dhpVerse}{87-88}
\label{dhp-87}\label{dhp-88}
Mit einem Bild der Freiheit als Ziel vor Augen\\ 
wendet sich der Weise von der Dunkelheit ab, \\ 
strebt dem Licht zu, lässt belanglose Sicherheit\\ 
zurück und sucht Freiheit von Anhaftung.\\ 
Eine solche Befreiung anzustreben\\ 
ist schwierig und selten,\\ 
dennoch wird der Weise sie suchen,\\ 
die Hindernisse überwinden,\\ 
Herz und Geist veredelnd. 
\end{dhpVerse}

\begin{dhpRefl}

Der Buddha führt uns mit Bildern das Ziel vor Augen, um uns zu ermutigen und
unsere Bemühungen zu unterstützen, das loszulassen, was uns hindert und
begrenzt. Wenn wir zu stark an solchen Bildern festhalten, können wir das Hier
und Jetzt aus den Augen verlieren – anstatt wirklich in die Praxis
einzutauchen, stellen wir sie uns nur vor. Wenn wir es nicht schaffen, das
Ziel mit Nachdruck zu verfolgen, können wir uns in der Zerstreuung von
Sinnesobjekten – angenehmen wie unangenehmen – verlieren. Wahre Freiheit zu
erlangen ist schwierig, aber bedenke, wie viel Leid uns bevorsteht, wenn wir
nicht praktizieren. Durch weise Reflexion gelingt es uns, die dunklen und
schwierigen Zeiten zu ertragen. Wenn das Licht dann zurückkehrt, schätzen wir
es und werden uns gewahr, wie wir die Wahrheit noch vollkommener lieben
können.

\end{dhpRefl}
