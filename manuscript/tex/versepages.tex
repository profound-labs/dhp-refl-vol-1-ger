%% == 1 ==

\begin{dhpVerse}{87-88}
\label{dhp-87}\label{dhp-88}
Mit einem Bild der Freiheit als Ziel vor Augen\\ 
wendet sich der Weise von der Dunkelheit ab, \\ 
strebt dem Licht zu, lässt belanglose Sicherheit\\ 
zurück und sucht Freiheit von Anhaftung.\\ 
Eine solche Befreiung anzustreben\\ 
ist schwierig und selten,\\ 
dennoch wird der Weise sie suchen,\\ 
die Hindernisse überwinden,\\ 
Herz und Geist veredelnd. 
\end{dhpVerse}

\begin{dhpRefl}

Der Buddha führt uns mit Bildern das Ziel vor Augen, um uns zu ermutigen und
unsere Bemühungen zu unterstützen, das loszulassen, was uns hindert und
begrenzt. Wenn wir zu stark an solchen Bildern festhalten, können wir das Hier
und Jetzt aus den Augen verlieren -- anstatt wirklich in die Praxis
einzutauchen, stellen wir sie uns nur vor. Wenn wir es nicht schaffen, das
Ziel mit Nachdruck zu verfolgen, können wir uns in der Zerstreuung von
Sinnesobjekten -- angenehmen wie unangenehmen -- verlieren. Wahre Freiheit zu
erlangen ist schwierig, aber bedenke, wie viel Leid uns bevorsteht, wenn wir
nicht praktizieren. Durch weise Reflexion gelingt es uns, die dunklen und
schwierigen Zeiten zu ertragen. Wenn das Licht dann zurückkehrt, schätzen wir
es und werden uns gewahr, wie wir die Wahrheit noch vollkommener lieben
können.

\end{dhpRefl}

%% == 2 ==

\begin{dhpVerse}{169}
\label{dhp-169}
Lebe dein Leben im Einklang mit dem Weg --\\ 
vermeide ein Leben voller Ablenkung.\\ 
Ein gut gelebtes Leben führt zu Zufriedenheit,\\ 
sowohl im Jetzt als auch in der Zukunft. 
\end{dhpVerse}

\begin{dhpRefl}

Mit einem zufriedenen Herzen als Fundament können wir uns den
Herausforderungen, die uns begegnen, stellen. Bisweilen müssen wir mutige
Krieger sein und uns mit den Kräften der Verblendung messen, um zu verhindern,
dass sie die Kontrolle über unseren Geist und unsere Herzen übernehmen. Zu
anderen Zeiten müssen wir wie Eltern sein, die sich um das schon vorhandene
Gute kümmern und es nähren. Flexibilität ist eine wichtige spirituelle Tugend.
Wenn wir die Schönheit erkennen, die einem zufriedenen Herzen innewohnt,
werden wir uns naturgemäß zu ihm hingezogen fühlen. Wir suchen Zerstreuung
nur, weil wir Zufriedenheit nicht kennen. Rechte Praxis setzt Energie frei,
die zuvor durch Zwanghaftigkeit verbraucht wurde. Dieselbe Energie kann sich
auch als Vitalität und Enthusiasmus äußern.

\end{dhpRefl}

%% == 3 ==

\begin{dhpVerse}{58-59}
\label{dhp-58}\label{dhp-59}
Wie eine wohlriechende und prachtvolle Lotosblume\\ 
aus einem Haufen Abfall wachsen kann,\\ 
so überscheint die Ausstrahlung eines wahren Schülers des Buddha\\ 
die dunklen Schatten der Ignoranz. 
\end{dhpVerse}

\begin{dhpRefl}

Wir neigen dazu, die Dinge, die wir an uns selbst und an anderen nicht mögen,
als Hemmnisse für unser Glücklichsein zu sehen. `Wie strahlend, wie erfüllt
würde ich mich fühlen, wie akzeptabel würde ich werden, wenn ich bloß all
diese Begrenzungen nicht hätte.' Doch alles, was wir erfahren, kann verwandelt
werden und unser Wachstum fördern, hin zu dem, was aus sich heraus zeitlos
schön ist. All die unfairen, unerwünschten und unbegründeten Dinge in unserem
Leben, all die Dinge, die uns widerstreben und die wir ablehnen, bilden den
Sumpf dessen, was als Abfall entsorgt wurde. Prachtvolle Lotosblumen wurzeln
und wachsen im `Unschönen'.

\end{dhpRefl}

%% == 4 ==

\begin{dhpVerse}{290}
\label{dhp-290}
Es ist Weisheit, die es ermöglicht\\ 
eine geringere Freude loszulassen,\\ 
um eine größere Freude anzustreben. 
\end{dhpVerse}

\begin{dhpRefl}

Die Filter unserer Vorlieben schränken auf tragische Weise unsere Wahrnehmung
ein. Wir möchten loslassen, was uns bindet, doch oft lässt uns unser Wille im
Stich. Weises Nachdenken kann den Willen unterstützen – es ist des Willens
bester Freund. Der Wille ist nicht dazu bestimmt, es alleine zu schaffen.
Dieser Vers ermutigt uns, darüber nachzudenken, wie das Loslassen unserer
Anhaftung an eine geringere Freude zum Erlangen einer größeren führen kann. In
unseren Anhaftungen versunken, sehen wir nur das, was wir verlieren könnten,
wenn wir loslassen. Weises Nachdenken hingegen bedeutet, dass wir nicht nur
sehen, was wir verlieren könnten, wenn wir loslassen, sondern auch das, was
wir gewinnen könnten. Weises Abwägen öffnet und erweitert unseren
Gesichtskreis und ermöglicht uns die Verfolgung unseres Ziels.

\end{dhpRefl}

%% == 5 ==

\begin{dhpVerse}{227-228}
\label{dhp-227}\label{dhp-228}
Schon seit alten Zeiten ist es so, dass jene,\\ 
die zu viel sprechen, kritisiert werden,\\ 
wie auch jene, die zu wenig sprechen,\\ 
und jene, die überhaupt nicht sprechen.\\ 
Ein jeder erfährt Kritik. Nie gab es,\\ 
noch gibt es, noch wird es jemals jemanden geben, \\ 
der nur Kritik oder nur Lob erfährt. 
\end{dhpVerse}

\begin{dhpRefl}

Was immer wir in diesem Leben tun, was immer wir sagen (oder nicht sagen), wir
können nicht verhindern, dass man uns kritisiert. Wie alle anderen wurde auch
der Buddha getadelt und kritisiert. Nur Lob zu suchen und Tadel zu fürchten,
ist sinnlos. Der einzige Tadel, mit dem wir uns wirklich befassen sollten, ist
jener, der von einem Weisen gegeben wird. Wenn wir von jemandem kritisiert
werden, der ein untadeliges Leben führt, ist es angemessen zu beherzigen, was
dieser sagt. Wenn uns aber jegliche Kritik verletzt, dann müssen wir genauer
hinschauen. Bedenke, dass Menschen Schmerz, den sie selbst nicht ertragen
können, in Form von Beschuldigungen aussenden. Sie zeigen ihre Verletztheit,
indem sie Fehler bei anderen suchen. Wenn wir aufmerksam sind und die
Fähigkeit besitzen uns selbst vollkommen zu akzeptieren, neigen wir nicht dazu
jemanden zu beschuldigen -- weder uns selbst noch andere.

\end{dhpRefl}

%% == 6 ==

\begin{dhpVerse}{328}
\label{dhp-328}
Solltest du einen guten Weggefährten finden,\\ 
voller Aufrichtigkeit und Weisheit,\\ 
wirst du alle Gefahren in fröhlicher\\ 
und fürsorglicher Gesellschaft überwinden. 
\end{dhpVerse}

\begin{dhpRefl}

Wie Wasser nimmt unser Geist die Form des Gefäßes an, in dem er sich befindet.
Der Lehrer ermuntert uns, darauf zu achten, mit wem wir Umgang pflegen. Die
Mangala-Sutta oder `Die Lehre über den großen Segen' besagt: “Meide die
Gesellschaft von Toren und suche die Gesellschaft von Weisen.” Wir sollten die
Gabe der Unterscheidung geschickt anwenden und uns bedachtsam darin üben,
Voreingenommenheit nicht mit weisem Erwägen zu verwechseln. Weises Erwägen ist
mitfühlend, gütig und am Wohl aller Wesen interessiert.

\end{dhpRefl}

%% == 7 ==

\begin{dhpVerse}{8}
\label{dhp-8}
So wie ein stürmischer Wind\\ 
einen Bergfelsen nicht bewegen kann,\\ 
so kann jemand, der die Wirklichkeit\\ 
des Körpers betrachtet, der Vertrauen\\ 
und Energie entwickelt, nicht von Māra\\ 
beeinflusst werden. 
\end{dhpVerse}

\begin{dhpRefl}

Māra ist die Manifestation der Verblendung, die treibende Kraft der
Verdrängung. Māra ist die Vermeidung der Wirklichkeit und drückt sich in Form
von Zwanghaftigkeit, Gefühllosigkeit und Ärger aus. Unser Kloster im
ländlichen Northumberland wird manchmal von kräftigen Winden heimgesucht, was
beängstigend sein kann, aber im Vergleich zu den Kräften Māras ziemlich
harmlos ist. Wenn wir den Kräften unserer achtlosen Gewohnheiten widerstehen
wollen, müssen wir unsere Übung der Achtsamkeit auf den Körper beständig und
mit Vertrauen und Energie weiterentwickeln. Es geht um die Fähigkeit und die
Bereitwilligkeit immer wieder auf unsere körperbezogene Praxis zurückzukommen,
uns an die Lebensweise zu erinnern, die zu selbstlosem Vertrauen führt und
unser Interesse aufrechterhält, zur Erkenntnis der Wahrheit zu gelangen. Das
hat die Kraft uns unerschütterlich zu machen.

\end{dhpRefl}

%% == 8 ==

\begin{dhpVerse}{104-105}
\label{dhp-104}\label{dhp-105}
Seiner selbst Herr zu sein,\\ 
ist der höchste Sieg -- viel mehr wert\\ 
als die Kontrolle über andere.\\ 
Es ist ein Sieg, den absolut niemand\\ 
beeinträchtigen oder wegnehmen kann. 
\end{dhpVerse}

\begin{dhpRefl}

Unerschütterlich im Zentrum unseres Lebens gefestigt, kann uns nichts aus der
Bahn werfen oder Grund zum Leiden geben. Das ist Furchtlosigkeit. Die Lehre
des Buddha deutet auf die Dinge hin, die Furchtlosigkeit verhindern:
gewohnheitsmäßige Selbstsucht, verblendetes Verlangen. Wenn wir das Licht der
Achtsamkeit genau auf diese Aktivität des Verlangens richten -- ohne auf
irgendeine Weise zu urteilen oder uns einzumischen -- erkennen wir allmählich,
dass jeder winzige Moment der Selbstbeherrschung zum Sieg der Selbstlosigkeit
führt.

\end{dhpRefl}

%% == 9 ==

\begin{dhpVerse}{268-269}
\label{dhp-268}\label{dhp-269}
Schweigen deutet nicht auf Weisheit hin,\\ 
wenn man unwissend und ungeschult ist.\\ 
Wie jemand, der eine Waage hält, wägt ein Weiser Dinge ab,\\ 
wohlbedachte und ungeschickte, und lernt sowohl\\ 
die innere als auch die äußere Welt kennen.\\ 
Deshalb wird der Weise als weise bezeichnet. 
\end{dhpVerse}

\begin{dhpRefl}

Der Buddha sprach von der Zufriedenheit und dem Nutzen, die vom Leben an einem
ruhigen und schönen Ort herrühren können. Sinnliche Reize zu beschränken kann
uns auf dem Weg zur Freiheit von Unwissenheit unterstützen. Er meinte damit
aber nicht, dass wir Stellung gegen die Sinne beziehen sollten. Ajahn Chah
sagte oft: “Wenn du nicht in der Stadt praktizieren kannst, kannst du auch
nicht im Wald praktizieren.” Und er sagte auch: “Wenn du nicht praktizieren
kannst, wenn du krank bist, dann kannst du auch nicht praktizieren, wenn du
gesund bist.” Mit anderen Worten, alles ist Praxis, einschließlich des
Gefühls, dass wir mit etwas nicht praktizieren können. Es ist Weisheit, die
diese Wahrheit erkennt.

\end{dhpRefl}

%% == 10 ==

\begin{dhpVerse}{172}
\label{dhp-172}
Es gibt jene, die  aus der\\ 
Achtlosigkeit erwacht sind.\\ 
Sie bringen Licht in die Welt,\\ 
wie der von Wolken befreite Mond. 
\end{dhpVerse}

\begin{dhpRefl}

Am Vollmondtag des Monats Asalha, vor über zweitausendfünfhundert Jahren,
legte der Buddha zum ersten Mal die Vier Edlen Wahrheiten dar. Die Herzen
jener, die diese Lehren hörten, waren mit der Freude vollkommenen Verstehens
erfüllt, so erfrischend und glänzend, wie der Mond, der hinter den Wolken
hervorleuchtet. Wir halten die Welt und unser Leiden für substantieller als
sie wirklich sind. Wenn wir unsere Aufmerksamkeit jedoch auf die wahren Gründe
der Unzufriedenheit lenken -- Verlangen begründet in Ignoranz -- erzeugen
diese Vier Veredelnden Wahrheiten eine Leuchtkraft, die die Wolken der
Verblendung zerstreut und die Welt des Leidens auflöst. Unsere Betrachtungen
erwecken uns aus der Achtlosigkeit. Augenblick für Augenblick tragen wir zum
Weiterdrehen des Dhamma-Rades bei.

\end{dhpRefl}

%% == 11 ==

\begin{dhpVerse}{78}
\label{dhp-78}
Suche nicht die Gemeinschaft\\ 
von irregeleiteten Freunden;\\ 
meide schlechte Gesellschaft.\\ 
Suche und genieße die Gemeinschaft\\ 
guter Freunde,\\ 
solcher, die Einsicht unterstützen. 
\end{dhpVerse}

\begin{dhpRefl}

Wie gehen wir mit unseren Freundschaften um? Wie behandeln uns unsere Freunde?
Wir möchten wahre Freundschaft mit jenen schützen und nähren, die unser
Bestreben, in der Wahrheit zu leben, unterstützen. Wir schätzen solche
Gefährten wahrhaftig; wir warten nicht, bis wir in Schwierigkeiten sind, bevor
wir sie wissen lassen, was sie uns bedeuten. Gute Freundschaft kann kultiviert
werden. Und während wir Achtsamkeit in unsere Beziehungen zu anderen bringen,
lasst uns unsere Aufmerksamkeit auch darauf richten, wie freundlich wir zu uns
selbst sind.

\end{dhpRefl}

%% == 12 ==

\begin{dhpVerse}{145}
\label{dhp-145}
Jene, die Kanäle bauen,\\ 
kanalisieren den Fluss des Wassers.\\ 
Bogenmacher fertigen Bögen.\\ 
Tischler bearbeiten Holz.\\ 
Jene, die sich der Güte verpflichten,\\ 
zähmen sich selbst. 
\end{dhpVerse}

\begin{dhpRefl}

Handwerker arbeiten in ihrem Handwerk. Die Bestrebung, die widerspenstige
Natur des Herzens zu zähmen, ist eine Kunst. Äußerlich betrachtet, mögen wir
uns in verwirrten Umständen befinden, aber erinnern wir uns, dass die
Kultivierung von Geschicklichkeit unsere primäre Aufgabe ist. Durch
vorsichtige und beständige Beobachtung lernen wir die Widerspenstigkeit des
Herzens kennen. „Das Leben sollte so nicht sein!“ Wir können lernen, auf
solche Gedanken nicht verurteilend zu reagieren. Unser
leidenschaftliches Verlangen, dass die Umstände anders sein sollten, als sie sind,
wird einfach als das erkannt, was es ist. Mit Hilfe dieser neuen Erkenntnis
finden wir zunehmendes Vertrauen und die Bereitschaft uns dem hinzugeben was
immer wir gerade tun -- von ganzem Herzen, mit unserem ganzen Körper. Wenn wir
weitergehen, wird die Aufgabe leichter; Dankbarkeit kommt auf, auch wenn die
Arbeit herausfordernd ist.

\end{dhpRefl}

%% == 13 ==

\begin{dhpVerse}{146}
\label{dhp-146}
Was soll das Gelächter?\\ 
Was soll das Vergnügen,\\ 
wenn die Welt in Flammen steht?\\ 
Da du von Dunkelheit umgeben bist,\\ 
solltest du nicht das Licht suchen? 
\end{dhpVerse}

\begin{dhpRefl}

Gewöhnlich wenden wir uns den Lehren erst dann zu, wenn wir vom Leben
enttäuscht worden sind. Es kann eine Erleichterung sein, festzustellen, dass
wir uns in unseren Bemühungen uns aus der Dunkelheit zu befreien, in der
Gesellschaft von Millionen von Mitmenschen befinden. Leiden ist die Natur der
nicht erwachten Menschheit. “Fühle dich nicht schlecht, wenn du leidest. Jeder
Mensch leidet,” würde uns Ajahn Chah sagen. Vor seiner Erleuchtung hat auch
der spätere Buddha gelitten. Der Unterschied ist, dass erleuchtete Wesen
wissen, dass Leiden keine Pflicht ist -- es ist nur eine von vielen in der
menschlichen Erfahrungswelt verfügbaren Optionen. Es gibt auch die Möglichkeit
im Licht des Nicht-Leidens zu verweilen.

\end{dhpRefl}

%% == 14 ==

\begin{dhpVerse}{239}
\label{dhp-239}
Ganz allmählich, Augenblick für Augenblick\\ 
entfernt der Weise seine eigenen Verunreinigungen,\\ 
so wie ein Goldschmied die Schlacke entfernt. 
\end{dhpVerse}

\begin{dhpRefl}

Kein noch so starker Wunsch, die Dinge möchten anders sein, gibt uns das,
wonach wir uns sehnen. Da wir uns nach dem reinen Gold ursprünglicher
Achtsamkeit sehnen, müssen wir uns auf das Feuer der Reinigung einlassen.
Dieser Vers unterweist uns darin, wie wir auf das Feuer achtgeben können: zu
viel Hitze -- wir bemühen uns zu sehr -- wenn wir achtlos erdulden, schaden
wir uns in unserer Praxis. Nicht genug Hitze -- Schwierigkeiten vermeidend --
es wird nicht zur Verbesserung unserer Praxis beitragen, wenn wir unserer
Vorliebe für Ruhe und Behaglichkeit folgen. So werden wir im Laufe der Jahre
bloß törichter. Unsere Gewohnheiten sind die Schlacke, und mit der
allmählichen Feinabstimmung unserer Bemühungen lernen wir loszulassen. Das
Ziel der ganzen Mühe ist die Verwirklichung des Zustandes leuchtender
Gegenwärtigkeit. Wir haben dann etwas unglaublich Wertvolles mit anderen zu
teilen.

\end{dhpRefl}

%% == 15 ==

\begin{dhpVerse}{178}
\label{dhp-178}
Besser als die ganze Welt zu beherrschen,\\ 
besser als in den Himmel zu kommen,\\ 
besser als die Herrschaft über das Universum,\\ 
ist die unumkehrbare Hingabe an den Weg. 
\end{dhpVerse}

\begin{dhpRefl}

Bedingungslose Freiheit: eine Qualität des Seins, unabhängig von
jeglichen Bedingungen. Unabhängig von den Umständen, ob erfreulich oder
unerfreulich, bleibt das Herz unbeschwert, strahlend, klarsichtig, empfindsam
und stark. Dies ist eine Hingabe, die unumkehrbar, unerschütterlich und real
ist, jenseits aller verblendeten Selbstbezogenheit. An diesen Punkt der
Entschlossenheit zu gelangen, erfordert ununterbrochene Beobachtung unserer
alten Gewohnheiten, wie zum Beispiel: die Vorliebe alles kontrollieren zu
wollen, die Abhängigkeit von flüchtigen Vergnügungen, Machtbesessenheit. Wir
arbeiten mit dem, was wir haben. Jedes Mal wenn wir erkennen, dass wir uns
haben ablenken lassen, fokussieren wir unsere Hingabe an den Weg von neuem.

\end{dhpRefl}

%% == 16 ==

\begin{dhpVerse}{76}
\label{dhp-76}
Segen kann nur von der Gemeinschaft\\ 
mit weisen und einsichtigen Menschen kommen,\\ 
die geschickt Ermahnung und Ratschlag geben,\\ 
als ob man zu verborgenen Schätzen geführt wird. 
\end{dhpVerse}

\begin{dhpRefl}

Es gibt zwei Wege unsere Praxis anzugehen: Ein Weg ist, anzunehmen, dass wir
noch etwas erlangen müssen, der andere, davon auszugehen, dass wir schon
haben, was wir brauchen -- aber noch nicht genau wissen, wo genau es sich
befindet. In diesem Vers gibt uns der Buddha das Bild eines verborgenen
Schatzes, der darauf wartet gefunden zu werden. Wir werden ihn nicht außerhalb
von uns selbst finden; vielmehr wohnt er unserem eigenen Herzen inne. Wir
erkennen ihn nicht, weil wir schon zu viel haben: zu viel Verlangen, zu viel
Widerstand. Es ist ein großer Segen einen weisen, einsichtigen Menschen zu
treffen, jemanden mit vollkommenem Mitgefühl, der bereit ist, uns anzuleiten
auf unserem Weg zur Befreiung von den Gewohnheiten die uns behindern. Ein
solcher Mensch kann uns durch trügerische Sümpfe des Zweifels oder über
bedrohliche Berge der Begierde führen, aber es kann nur gewinnbringend sein,
wenn wir sein Beispiel ehren und den Weg bis zum Ende gehen.

\end{dhpRefl}

%% == 17 ==

\begin{dhpVerse}{1}
\label{dhp-1}
Alle Gemütszustände werden durch den Geist bestimmt.\\ 
Der Geist geht voraus. Genauso, wie\\ 
das Rad des Ochsenkarrens der Hufspur des Zugtieres folgt,\\ 
so wird Leiden mit Gewissheit folgen, wenn wir impulsiv aus\\ 
einer unreinen Geistesverfassung heraus sprechen oder handeln. 
\end{dhpVerse}

\begin{dhpRefl}

Zuviel zu denken kann die spirituelle Praxis erschweren; nicht genug zu denken
kann uns einschränken. Hier wird uns ein Beispiel von weise gelenktem
Nachdenken gegeben. Es hilft uns, uns nicht als Opfer unserer Umstände zu
sehen und zu verstehen, dass die Absicht hinter unseren Handlungen und unserer
Sprache unsere geistige Verfassung bedingt -- es gibt uns Kraft wirklich etwas
zu verändern. Indem wir diese Lehre als Untersuchungsfeld akzeptieren, lernen
wir den Wert schätzen, der im Üben achtsamer Zurückhaltung liegt und finden
neue Zuversicht und Kraft. Unser Herz/Geist führt uns auf dem Weg.

\end{dhpRefl}

%% == 18 ==

\begin{dhpVerse}{118}
\label{dhp-118}
Hat man eine gute Tat vollbracht,\\ 
so ist es gut sie zu wiederholen.\\ 
Genieße die Freude an diese Erinnerung.\\ 
Die Frucht von Güte ist Zufriedenheit. 
\end{dhpVerse}

\begin{dhpRefl}

Der Dhamma ermutigt uns, die Erinnerung an förderliche Taten, seien sie
geistiger, sprachlicher oder körperlicher Art, wachzuhalten und in der Freude
der Erinnerung zu verweilen. Wenn wir achtsam sind, ist das Risiko gering,
dass wir allzu selbstgefällig werden. Desgleichen können wir, wenn wir
weiseses Reflektieren entwickelt haben, unsere Fehler und Mängel betrachten,
ohne unsere Güte aus den Augen zu verlieren. Dass wir unsere Gewohnheiten
beobachten können, zeigt, dass wir mehr sind als diese. Was ist es, das
beobachtet? Es ist unsere Zuflucht -- Gegenwärtigkeit -- der Weg aus dem
Leiden heraus. Mit genau dieser Gegenwärtigkeit können wir frei in der Freude
an die Erinnerung an Güte verweilen. Und was für eine Erleichterung ist es, zu
erkennen, dass wir Fehler machen und aus ihnen lernen können.

\end{dhpRefl}

%% == 19 ==

\begin{dhpVerse}{348}
\label{dhp-348}
Lass los, was vor dir ist,\\ 
lass los, was schon vergangen ist,\\ 
und lass los von dem, was dazwischen ist.\\ 
Mit einem Herzen,\\ 
dass sich nirgendwo festhält,\\ 
kommst du am Ort jenseits\\ 
allen Leidens an. 
\end{dhpVerse}

\begin{dhpRefl}

Es fällt uns leicht, uns in Ideen an die Zukunft zu verlieren. Es fällt uns
leicht, in Erinnerungen an die Vergangenheit zu schwelgen. Und gerne verlieren
wir uns auch in Erlebnissen im Hier und Jetzt. Wenn wir uns verlieren, leiden
wir. Jedoch ist Leiden nicht endgültig; es gibt Ursachen für das Leiden und es
gibt Freiheit von diesen Ursachen. Um Freiheit zu erlangen, ist es
erforderlich, dass wir vertraute und liebgewonnene Gewohnheiten hinter uns
lassen. Wir hören die Anweisung `loslassen' und sie klingt vielleicht, als
sage man uns, wir müssten etwas loswerden oder wir seien fehlerhaft, so wie
wir sind. Doch wenn wir die Kraft Rechter Achtsamkeit erkennen, sehen wir,
dass Loslassen einfach passiert. In diesem Vers begegnet uns der Buddha am Ort
unseres Leidens und zeigt uns den Weg aus ihm heraus.

\end{dhpRefl}

%% == 20 ==

\begin{dhpVerse}{387}
\label{dhp-387}
Die Sonne scheint am Tag,\\ 
der Mond scheint bei Nacht.\\ 
Aber sowohl den ganzen Tag\\ 
als auch die ganze Nacht\\ 
scheint des Buddhas\\ 
glorreiche Pracht. 
\end{dhpVerse}

\begin{dhpRefl}

Wenn Achtsamkeit voll entfaltet ist, dann sind Schönheit und Klarheit da und
die Möglichkeit das Verstehen zu vertiefen. Wenn ununterbrochene Achtsamkeit
vorhanden ist, dann ist dauerhafte Klarheit da. Den ganzen Tag und die ganze
Nacht lang scheint das Herz der Achtsamkeit in glorreicher Pracht.

\end{dhpRefl}

%% == 21 ==

\begin{dhpVerse}{122}
\label{dhp-122}
Ignoriere nicht den Effekt Rechter Handlung,\\ 
indem du sagst: “Das wird zu nichts führen.” Gerade so,\\ 
wie das gleichmäßige Fallen von Regentropfen\\ 
einen Wasserkrug füllt, so wird der Weise\\ 
mit der Zeit reichlich mit Güte versehen. 
\end{dhpVerse}

\begin{dhpRefl}

“Größer ist besser.” “Je mehr desto besser.” Solche Vorstellungen können nach
und nach weise hinterfragt werden. Vertraue darauf, dass auch der winzigste
Moment, in dem du gegenwärtig bist, Bedeutung hat. Selbst das kleinste Bemühen
sich zu erinnern, wieder zurückzukommen und mit nicht bewertender
Gegenwärtigkeit neu zu beginnen, macht einen Unterschied. Ein solches Bemühen
ist nie verschwendet. Eines Tages stellen wir fest, dass wir nicht länger von
dem beeindruckt sind, was uns vorher fesselte. Anstatt auf etwas zu reagieren,
was uns früher aus der Fassung gebracht hätte, können wir loslassen. Weisheit
kennt den Weg zu wahrer Güte.

\end{dhpRefl}

%% == 22 ==

\begin{dhpVerse}{134}
\label{dhp-134}
Spricht man derb zu dir,\\ 
werde still\\ 
wie ein gesprungener Gong;\\ 
Vergebung ist ein Zeichen von Freiheit. 
\end{dhpVerse}

\begin{dhpRefl}

Wenn wir unberechtigter Kritik ausgesetzt sind, kann es schwer sein, das
aufflammende Feuer zu beherrschen. Starke Gefühle zu verdrängen ist nicht
hilfreich. Praktizieren heißt, den Ort in uns zu finden, an dem wir fühlen was
wir fühlen, ohne zu diesen Gefühlen zu `werden'. Das ist eine besondere
Fähigkeit. Sei auf der Hut vor Stimmen, die dir predigen wollen: “Du solltest
nicht so sein, du solltest es mittlerweile besser wissen.” Wir erkennen die
Dinge an, wie sie in diesem Moment sind. Akzeptiere die gegenwärtige Realität
in dem Wissen, dass sie so ist wie sie ist; schwelge nicht in ihr und lehne
sie auch nicht ab. So kann die Energie deiner Leidenschaften zum Brennstoff
für den Reinigungsprozess werden, der Verunreinigungen ausbrennt, anstatt dich
selbst mit Selbstkritik aufzuzehren.

\end{dhpRefl}

%% == 23 ==

\begin{dhpVerse}{130}
\label{dhp-130}
Hat man Einfühlungsvermögen für andere,\\ 
sieht man, dass alle Wesen\\ 
das Leben lieben und den Tod fürchten.\\ 
Wenn man dies weiß, bedroht man niemanden,\\ 
noch verursacht man Bedrohung. 
\end{dhpVerse}

\begin{dhpRefl}

Anderen gegenüber offen zu sein, kommt uns selbst zugute. Indem wir unser Herz
und unseren Geist untersuchen, fördern wir einander. Dhamma lehrt uns auf
beiden Ebenen zu praktizieren. Auf der einen Ebene kultivieren wir Einsicht,
was tief verankerte Ansichten vom `Selbst' und den `Anderen' auflöst. Das ist
unsere formale Praxis. Währenddessen fordert unsere Alltagspraxis uns dazu
auf, die relative Gültigkeit von Ansichten wie derjenigen vom `Selbst' und den
`Anderen' zu akzeptieren, um diese zur Kultivierung von Mitgefühl und Güte zu
verwenden. Einfühlungsvermögen geht einher mit Einsicht und hilft die Praxis
im Gleichgewicht zu halten.

\end{dhpRefl}

%% == 24 ==

\begin{dhpVerse}{262-263}
\label{dhp-262}\label{dhp-263}
Jene, die neidisch, geizig und manipulierend sind,\\ 
bleiben unsympathisch\\ 
trotz guten Aussehens und gewandter Sprache.\\ 
Aber jene, die sich selbst von ihren Fehlern befreiten\\ 
und Weisheit kultivierten, sind wirklich attraktiv. 
\end{dhpVerse}

\begin{dhpRefl}

Wir müssen uns daran erinnern, dass die Dinge nicht unbedingt so `sind', wie
sie erscheinen. Wenn wir jemandem mit dem Herzen zuhören statt nur mit den
Ohren, werden wir etwas ganz anderes hören. Wenn wir andere von einem ruhigen,
zentrierten Ort tief in unserem Innern betrachten, sehen wir viel mehr als
das, was unsere Augen sehen.

\end{dhpRefl}

%% == 25 ==

\begin{dhpVerse}{162}
\label{dhp-162}
Übelgesinnte handeln sich selbst gegenüber\\ 
genauso wie ihre eigenen schlimmsten Feinde.\\ 
Sie sind wie Schlingpflanzen, die die Bäume ersticken,\\ 
die sie unterstützen. 
\end{dhpVerse}

\begin{dhpRefl}

Dieser Vers bezieht sich auf einen Mönch, der dreimal versuchte den Buddha zu
töten und dessen böse Taten schließlich seinen eigenen Tod herbeiführten. Wenn
wir unseren Herzenswunsch nach Wahrheit verleugnen, fallen wir langsam aber
sicher vom Licht der Wahrheit ab und versinken in der Finsternis.
Schlingpflanzen ranken sich um stattliche, ausgewachsene Bäume und ersticken
diese manchmal, so dass sie absterben. Wir können Zuflucht zum Buddha nehmen
und dennoch von Wutausbrüchen überwältigt werden. Stunden, Tage oder sogar
Jahre können vergehen, in denen wir unsere schädlichen Taten und Worte
rechtfertigen. Wenn wir die Wahrheit in Bezug auf unsere Taten erkennen,
empfinden wir ein gesundes Maß an Reue; wir möchten wirklich von ihnen
ablassen. Rechtes Handeln ist die natürliche Konsequenz.

\end{dhpRefl}

%% == 26 ==

\begin{dhpVerse}{184}
\label{dhp-184}
Der Welt-Entsager unterdrückt niemanden.\\ 
Geduldiges Ertragen ist die höchste Askese.\\ 
Endgültige Befreiung, sagen die Buddhas,\\ 
ist das höchste Ziel. 
\end{dhpVerse}

\begin{dhpRefl}

Der Buddha ermutigt uns, nicht zu vergessen, wie wir mit anderen umgehen,
während wir auf unserem Weg zum Ziel fortschreiten. Die Kultivierung von
\emph{Metta} lehrt uns, alle Wesen in unser gütiges Herz einzubeziehen.
Manchmal kommt es vor, dass wir zwar freundlich zu anderen sein können, nicht
jedoch zu uns selbst. Wenn wir das schwierig finden, üben wir grenzenlose Güte
mit dem oder der Leidenden, nämlich uns selbst. Schichten von
Selbstverurteilung, Selbsthass oder Selbstablehnung bedingen, dass wir uns in
Geduld und Ausdauer üben müssen -- was nicht mit bitterem Ertragen zu
verwechseln ist. Geduld ist eine unerlässliche Eigenschaft für jene, die den
Weg beschreiten. Und es ist völlig unmöglich diese unverzichtbare Tugend zu
kultivieren, wenn es uns gut geht. Wenn es uns also schlecht geht, denkt
daran, dass dies der perfekte Ort ist, der einzige Ort, an dem wir diese
tiefgründige Qualität entwickeln können.

\end{dhpRefl}

%% == 27 ==

\begin{dhpVerse}{6}
\label{dhp-6}
Jene, die streitsüchtig sind,\\ 
haben vergessen,\\ 
dass wir alle sterben müssen;\\ 
für den Weisen,\\ 
der diese Tatsache bedenkt,\\ 
gibt es kein Streiten. 
\end{dhpVerse}

\begin{dhpRefl}

Es ist in Ordnung über den Tod nachzudenken und über die Tatsache zu sprechen,
dass wir alle sterben müssen. Es ist sogar weise, das zu tun. Wenn wir uns
unsere Sterblichkeit nicht bewusst machen, bleibt die Angst davor im
Unbewussten verborgen. Die Vermeidung der Realität führt zu verminderter
Lebendigkeit und zunehmender Verwirrung. Wenn wir hingegen Achtsamkeit für das
Thema Tod (\emph{maranasati}) kultivieren, stellen wir fest, dass wir in dem
Maße, wie wir Realität zulassen, auch an Klarheit und Zufriedenheit gewinnen.
Das ist die Zufriedenheit, die zu tieferem Frieden führt.

\end{dhpRefl}

%% == 28 ==

\begin{dhpVerse}{2}
\label{dhp-2}
Alle Gemütszustände werden durch den Geist bestimmt.\\ 
Der Geist geht voran.\\ 
So sicher, wie unser Schatten uns niemals verlässt,\\ 
so wird uns Wohlsein folgen,\\ 
wenn wir aus einem reinen Gemütszustand heraus\\ 
sprechen oder handeln. 
\end{dhpVerse}

\begin{dhpRefl}

Hast du jemals probiert vor deinem eigenen Schatten wegzulaufen? So sehr du es
auch versuchst, der Schatten verlässt dich nie. Jede Handlung, die einer
reinen Motivation entspringt, ob körperlicher, sprachlicher oder geistiger
Natur, wird tatsächlich zu größerem Wohlbefinden führen. Es lohnt sich, sich
daran zu erinnern, wenn wir Gutes tun wollen und dann im Begriff sind unseren
Vorsatz zu ändern, weil wir glauben, dass sich die Mühe nicht lohnt. So klein
und scheinbar unbedeutend die Tat auch sein mag, Freude wird folgen. Sie ist
der Mühe wert.

\end{dhpRefl}

%% == 29 ==

\begin{dhpVerse}{3-4}
\label{dhp-3}\label{dhp-4}
Wenn wir an Gedanken festhalten wie:\\ 
“Sie missbrauchten mich, behandelten mich schlecht,\\ 
belästigten mich, beraubten mich,”\\ 
halten wir Hass am Leben. 

Wenn wir uns zutiefst von solchen Gedanken loslösen wie:\\ 
“Sie missbrauchten mich, behandelten mich schlecht,\\ 
belästigten mich, beraubten mich,”\\ 
wird Hass überwunden. 
\end{dhpVerse}

\begin{dhpRefl}

Auf verschiedene Art und Weise widerfahren jedem Menschen Ungerechtigkeiten im
Leben. Manchmal steckt der Schmerz tief und dauert viele Jahre an. Die
Dhamma-Lehren betonen nicht so sehr den Schmerz, sondern unsere Beziehung zu
selbigem. Solange wir von Hass und Widerstand besessen sind, ist unsere
Intelligenz beeinträchtigt. Auch wenn es vielleicht nötig wäre zu handeln,
können wir nicht wissen, was die richtige Handlung wäre, wenn unser Herz nicht
frei von Hass ist. Der Buddha tritt dafür ein, dass wir uns von Gedanken des
Hasses befreien. Man braucht Stärke, Geduld und Willenskraft um loszulassen.
Wir lassen nicht los, weil uns jemand sagt, dass wir es tun sollten. Wir
lassen los, weil wir die Konsequenzen des Festhaltens verstehen.

\end{dhpRefl}

%% == 30 ==

\begin{dhpVerse}{377}
\label{dhp-377}
So wie verwelkte Blüten\\ 
von einem Jasminstrauch fallen,\\ 
lass Gier und Hass abfallen. 
\end{dhpVerse}

\begin{dhpRefl}

In der Zeit, als Ajahn Chah die Hampstead Vihara in London besuchte, fühlten
sich die Mönche durch den Lärm gestört, der aus der gegenüberliegenden Kneipe
kam. Ajahn Chah sagte ihnen, dass ihr Leiden daher komme, dass sie ihre
Aufmerksamkeit auf das Geräusch richteten. Geräusch ist einfach was es ist.
Leiden entsteht nur, wenn wir `hinausgehen' und etwas hinzufügen. Wenn wir
unseren Anteil an der Schaffung von Problemen erkennen, beginnen wir,
Schwierigkeiten aus einem anderen Blickwinkel zu betrachten. Anstatt andere zu
beschuldigen, `sehen' wir einfach, was wir im Moment tun. Lasst uns mit dem
Hass keinen Kampf beginnen; in der Ausübung vorsichtiger Zurückhaltung und
klugen Betrachtens lassen wir ihn `abfallen'. Anfänglich sehen wir das nur,
nachdem wir bereits reagiert und Leiden verursacht haben. Mit zunehmender
Übung erfassen wir es schneller. Eines Tages werden wir uns in dem Moment
ertappen, in dem wir dabei sind ein Problem zu schaffen.

\end{dhpRefl}

%% == 31 ==

\begin{dhpVerse}{401}
\label{dhp-401}
So wie Wasser von einem Lotosblatt abgleitet,\\ 
so haften Sinnesbegierden nicht\\ 
an einem großen Wesen. 
\end{dhpVerse}

\begin{dhpRefl}

Große Wesen sind groß, weil sie frei von Behinderungen in ihrer Beziehung zum
Leben sind. Wir sind nicht so groß, weil wir uns in Gefühle verfangen und ein
Problem aus dem Leben machen. Wir schaffen Behinderungen durch die Art und
Weise, wie wir mit den acht weltlichen Dhammas umgehen: Lob und Tadel, Gewinn
und Verlust, Freude und Schmerz, Beliebtheit und Unbeliebtheit. In unserer
Verblendung verhalten wir uns unachtsam gegenüber diesen weltlichen Strömungen
-- wir verlieren uns in dem, was wir mögen, und lehnen ab, was wir nicht
mögen. Hingegen sieht Weisheit einfach die Realität der Sinneswelt. Sie kennt
den Raum, in dem alle Erlebnisse entstehen und vergehen. Ein solches
Verständnis bedeutet, dass ein großes Wesen nicht einmal den Versuch
unternehmen muss loszulassen; jede Neigung festzuhalten fällt ganz von alleine
weg. Er oder sie erlebt Sinnesfreuden, fügt aber nichts hinzu und nimmt nichts
weg.

\end{dhpRefl}

%% == 32 ==

\begin{dhpVerse}{5}
\label{dhp-5}
Niemals wird Hass durch Hass besiegt,\\ 
nur durch Bereitschaft\\ 
zu liebender Güte allein.\\ 
Dies ist ein ewiges Gesetz. 
\end{dhpVerse}

\begin{dhpRefl}

Hass `allein durch liebende Güte' zu besiegen, kann wie ein unerreichbares
Ideal scheinen. Wir können uns selbst als zu begrenzt empfinden (und so dazu
tendieren uns selbst zu entschuldigen) -- oder aber die Idee als unrealistisch
abtun. Gewiss sagt ein Teil von uns `Ja'. Ein anderer Teil von uns sagt aber
möglicherweise: `Ja, aber was wäre, wenn… Für den Buddha stand dies nicht in
Frage; zu hassen funktioniert einfach nie. Was ist diese `liebende Güte'?
Uneingeschränkte Empfänglichkeit zu entwickeln? Und die Wahrnehmungen, die wir
von denen haben, die Hass in uns auslösen: Wo existieren diese Wahrnehmungen?
Liebende Güte hat potentiell die Fähigkeit alles mit Achtsamkeit zu empfangen,
vollständig, ganz und gar, ja sogar warmherzig. Dies ist ein ewiges Gesetz.

\end{dhpRefl}

%% == 33 ==

\begin{dhpVerse}{142}
\label{dhp-142}
Eine extravagante äußere Erscheinung\\ 
steht der Freiheit nicht im Weg.\\ 
Ein friedvolles Herz zu haben,\\ 
rein, zufrieden, erwacht und tadellos,\\ 
zeichnet ein edles Wesen aus. 
\end{dhpVerse}

\begin{dhpRefl}

Es kommt nicht wirklich auf die äußere Form an. Der Buddha hat immer auf das
Herz hingewiesen als den Ort, auf den wir uns fokussieren sollten. Er hob dies
hervor, weil wir es leicht vergessen, weil wir zu beschäftigt damit sind, wie
die Dinge erscheinen. Dieser Vers handelt von einem betrunkenen Haushälter,
der aufgrund einer herben Enttäuschung verzweifelt und schockiert war. Die
Lehren des Buddha wiesen ihn ganz unmittelbar auf das hin, was wirklich
wichtig ist. Das transformierte seine Verzweiflung; anstatt verrückt zu
werden, verwirklichte er vollkommenen, bedingungslosen Frieden. Wenn der Fokus
unserer Praxis bloß auf Formen und Erscheinungen ausgerichtet ist, dann
schadet das dem Geist. Klammern wir uns zum Beispiel zu fest an eine Erklärung
über die fünf ethischen Übungsregeln, kann uns das daran hindern, die Absicht
hinter der Tat zu erkennen. Die Form der Regeln ist dazu da, dass wir unsere
Motivation besser wahrnehmen können. Wenn wir die Regeln auf die richtige Art
und Weise einhalten, dann besteht die Möglichkeit, dass sie ihren wahren Zweck
erfüllen. Sein Leben richtig zu führen ist edles Sein.

\end{dhpRefl}

%% == 34 ==

\begin{dhpVerse}{95}
\label{dhp-95}
Da sind jene, die entdecken,\\ 
dass sie verwirrte Reaktionen\\ 
hinter sich lassen können\\ 
und geduldig wie die Erde werden;\\ 
ungerührt von Ärger,\\ 
unerschütterlich wie eine Säule,\\ 
ruhig wie ein klarer und stiller Teich. 
\end{dhpVerse}

\begin{dhpRefl}

Der Buddha lebte in dieser Welt, so wie wir auch. Und trotz all des Chaos
verwirklichte er einen Zustand der Gelassenheit -- er wurde `geduldig wie die
Erde'. Was auch immer auf sie gegossen, auf ihr verbrannt wird oder auf ihr
geschieht, die Erde bleibt einfach, was sie ist; sie tut, was sie tut.
Geduldiges Ertragen ist kein Zeichen von Schwäche; es ist bestärkend und
gutherzig. Durch geduldiges Ertragen entdecken wir die Fähigkeit unserem
gegenwärtigen Erleben zu erlauben hier zu sein, so wie es ist, bis wir gelernt
haben, was wir lernen müssen.

\end{dhpRefl}

%% == 35 ==

\begin{dhpVerse}{256}
\label{dhp-256}
Willkürliche Entscheidungen zu treffen\\ 
kann nicht als gerecht bezeichnet werden.\\ 
Der Weise trifft Entscheidungen,\\ 
nachdem er das Für und Wider abgewogen hat. 
\end{dhpVerse}

\begin{dhpRefl}

Du stehst unter Druck eine Entscheidung zu treffen. Ist es trotzdem möglich
ruhig und gelassen zu bleiben, wenn andere wollen, dass du zu ihren Gunsten
entscheidest? Können wir frei von Vorurteilen bleiben und zu einer gerechten
Entscheidung kommen? Wie vertreten wir unsere Ansichten? Eine starke Meinung zu
haben, kann sich großartig anfühlen; es kann den Anschein von Selbstvertrauen
geben. Doch das ist das Wesen des Fundamentalismus -- ebenso wie vereinfachte
Antworten auf komplexe Fragen zu geben. Starre Ansichten und allzu simple
Antworten sind keine Merkmale eines offenen Geistes, eines Geistes, der alle
Aspekte eines Dilemmas berücksichtigen kann. Man braucht gewöhnlich Zeit, um
zu einer ausgewogenen und gründlich durchdachten Ansicht zu gelangen. Auch die
Fähigkeit von einem Ort innerer Ruhe zuhören zu können, ist dafür
erforderlich. Wenn wir im Geiste bereits unsere Entgegnung vorbereiten, hören
wir nicht wirklich zu.

\end{dhpRefl}

%% == 36 ==

\begin{dhpVerse}{120}
\label{dhp-120}
Sogar jene, die ein tugendhaftes Leben führen,\\ 
können Leiden erfahren, solange ihre Taten\\ 
noch keine direkten Früchte getragen haben.\\ 
Doch, wenn die Früchte ihrer Taten reifen, können\\ 
die freudvollen Folgen nicht vermieden werden. 
\end{dhpVerse}

\begin{dhpRefl}

Wahren Prinzipien zu vertrauen ist nicht immer einfach. Populären Meinungen
anzuhängen ist nicht immer weise. Wenn wir uns nicht von Zweifeln oder bloßen
Spekulationen verleiten lassen, sehen wir, wie uns unsere tiefsten Fragen über
die Realität zur Einsicht führen. Vertrauen ist eine transformierende Kraft
ebenso wie Geduld. Es ist nahezu unvorstellbar, wie eine kleine Eichel zu
einer kräftigen Eiche heranwachsen kann, aber es geschieht. Wir brauchen
Vertrauen. Wir brauchen Geduld. Wie Bäume können diese transformierenden
Kräfte kultiviert werden und uns zu dem inspirieren, was für sich großartig
und naturgemäß frei von allem Leiden~ist.

\end{dhpRefl}

%% == 37 ==

\begin{dhpVerse}{91}
\label{dhp-91}
Wachsam gegenüber den Bedürfnissen der Reise,\\ 
gleiten jene, die auf dem Weg der Achtsamkeit sind,\\
weiter, wie Schwäne,\\
ihre früheren Rastplätze hinter sich lassend. 
\end{dhpVerse}

\begin{dhpRefl}

Es ist Zeit weiter zu gehen. Wir mögen dies auf einer bestimmten Ebene
erkennen, doch ein anderer Teil von uns zögert. Wenn wir versuchen
loszulassen, weil wir glauben, wir sollten es tun, werden wir Widerstand
spüren, anstatt neue Inspiration zu schöpfen. Aber auf Enttäuschung zu stoßen
ist ebenfalls in Ordnung, wenn wir auf die wahren Bedürfnisse unseres Herzens
achten. Unser tiefstes Bedürfnis ist frei von Ignoranz zu sein. Es gibt keinen
Ort, an dem wir uns wirklich niederlassen können, bis wir vollständig erkannt
haben, wer und was wir wirklich sind. Dieser Weg der Achtsamkeit hat die Kraft
absolut alles in Verständnis umzuwandeln. Schwäne gleiten weiter -- ohne
Widerstand.

\end{dhpRefl}

%% == 38 ==

\begin{dhpVerse}{223}
\label{dhp-223}
Transformiere Ärger durch Liebenswürdigkeit\\ 
und Böses durch Gutes,\\ 
Niedertracht durch Großzügigkeit\\ 
und Täuschung durch Ehrlichkeit.
\end{dhpVerse}

\begin{dhpRefl}

Wenn wir frieren, finden wir einen Weg uns aufzuwärmen -- wir setzen uns nicht
weiter der Kälte aus. Wenn wir hungrig sind, essen wir -- anstatt uns weiter
die Nahrung zu entziehen. Wenn wir ärgerlich sind, bekämpfen wir den Ärger
nicht mit Wut -- vielmehr versuchen wir nett zu dem Wesen zu sein, das unter
dem Ärger leidet. Wenn wir böse Taten mitansehen, erzeugen wir aufrichtige
Güte und zügeln jedweden Impuls, den Täter durch bloße Verurteilung
abzulehnen. Den Menschen, die glauben, dass selbstbezogene Gemeinheit der Weg
zur Zufriedenheit sei, begegnen wir mit Großzügigkeit. Und denen, die mit
doppelter Zunge sprechen, sagen wir die Wahrheit. Das mag nicht einfach sein.
Aber es ist der Weg der~Transformation.

\end{dhpRefl}

%% == 39 ==

\begin{dhpVerse}{258}
\label{dhp-258}
Jene, die viel reden, sind nicht\\
notwendigerweise im Besitz von Weisheit.\\
Der Weise kann dadurch erkannt werden,\\
dass er mit dem Leben im Einklang,\\
frei von Feindseligkeit und Angst ist.
\end{dhpVerse}

\begin{dhpRefl}

Wenn man sich anstrengt, friedvoll und weise zu werden, gießt man oft Öl ins
Feuer, während sich die Dinge beruhigen, wenn man es den Aktivitäten des
Geistes und des Herzens behutsam erlaubt, auf natürliche Weise auszuklingen.
Weisheit wird nicht dadurch entwickelt, dass wir der Gewohnheit folgen, das zu
suchen, was wir mögen. Wenn Schmerz und Enttäuschung aufkommen, können wir
lernen, ihnen mit einer Qualität von Bereitschaft und Aufmerksamkeit
entgegenzutreten, die zu Verständnis führt. Wenn wir Freude und Entzücken
erleben, können wir uns durch diese Erfahrung erfrischt und erneuert fühlen,
ohne uns darin zu verlieren. Feindseligkeit und Angst stellen nicht unbedingt
Hindernisse dar. Sie können uns etwas über die Möglichkeit einer freieren und
weiseren Lebensweise lehren.

\end{dhpRefl}

%% == 40 ==

\begin{dhpVerse}{82}
\label{dhp-82}
Beim Hören wahrer Lehren werden\\ 
die Herzen jener, die empfänglich sind,\\ 
ruhig wie ein See, tief, klar und still. 
\end{dhpVerse}

\begin{dhpRefl}

Wir denken vielleicht: `Wäre der Buddha heute hier, um mich zu lehren, könnte
auch ich erleuchtet werden.’ Doch als der Buddha lebte, gab es viele, die ihn
trafen, ihn hörten, mit ihm lebten, und dennoch weder ihn noch die Wahrheit
erkannten, auf die er hinwies. Am Ende seines Lebens fragte einer der Mönche
den Buddha, wer seinen Platz einnehmen würde, wenn er gegangen sei, worauf er
erwiderte, dass der Dhamma seinen Platz einnehmen würde. Bei einer früheren
Gelegenheit lehrte er, dass den Dhamma zu sehen dasselbe sei, wie den Buddha
zu sehen. Den Dhamma zu hören, bedeutet vollkommen achtsam und weise das zu
betrachten, was genau jetzt passiert. Es muss uns also nicht leid tun, dass
wir nicht einen Vortrag vom Buddha selbst hören können. Wir lernen
empfänglicher für das Leben zu sein.

\end{dhpRefl}

%% == 41 ==

\begin{dhpVerse}{173}
\label{dhp-173}
Jemand, der alte und achtlose Gewohnheiten\\ 
in frische und tugendhafte Taten verwandelt,\\ 
bringt Licht in die Welt,\\ 
wie der von Wolken befreite Mond. 
\end{dhpVerse}

\begin{dhpRefl}

Die Dunkelheit der Welt entspringt den Gewohnheiten der Vermeidung und des
übermäßigen Genusses. Heilsame Handlungen machen das schon vorhandene Licht
sichtbar. Betrachtet man alte, ausgetretene Wege auf die rechte Art und zur
rechten Zeit, führt das zur Verwandlung durch Aufmerksamkeit: Hier und Jetzt,
frei von Urteilen, in Körper und Geist verankert. Wir müssen nichts loswerden.
Wir haben nichts Schlechtes in uns. Schmerzhafte Erinnerungen oder
schmerzhafte Empfindungen müssen nicht zu Leiden werden. Mit Rechter
Achtsamkeit führt alles zu immer mehr Licht.

\end{dhpRefl}

%% == 42 ==

\begin{dhpVerse}{186-187}
\label{dhp-186}\label{dhp-187}
Weder durch großen Reichtum findet man Zufriedenheit,\\ 
noch in Sinnesvergnügen, ob grob oder fein.\\ 
Aber in der Auslöschung von Verlangen\\ 
ist Freude zu finden für einem Schüler des Buddha. 
\end{dhpVerse}

\begin{dhpRefl}

Ist das, was ich in meinem Leben tue, das, wonach ich suche? Gelegentlich
gelingt es uns, unsere Wünsche zu befriedigen, doch dauerhafte Freude erwächst
nur dann, wenn wir uns nicht mehr vom Reiz des Verlangens irritieren lassen.
Wenn wir uns zur rechten Zeit die richtigen Fragen stellen, wird ein völlig
anderer Blickwinkel offenbar: Der Weg der Zufriedenheit führt uns nach innen
anstatt nach außen. Besser als einem Impuls des `Mehr-Wollens' zu folgen, ist
es, das Verlangen direkt zu betrachten. Dann verschwindet für einen Moment der
Stachel des Verlangens. So haben wir ein bisschen mehr darüber gelernt, was es
bedeutet ein Schüler des Buddha zu sein.

\end{dhpRefl}

%% == 43 ==

\begin{dhpVerse}{14}
\label{dhp-14}
So wie Regen ein gut gedecktes Strohdach\\ 
nicht durchdringen kann,\\ 
können die Leidenschaften nicht\\ 
in ein gut trainiertes Herz eindringen. 
\end{dhpVerse}

\begin{dhpRefl}

Wir handeln, um uns vor den Elementen Wind und Regen zu schützen. Dem Handeln
geht weise Reflexion voraus, damit wir uns nicht ungezügelter Leidenschaft
überlassen. Das ist das Training des Herzens. Geschickt mit dem Schaden
umzugehen, der durch Begierde und Übelwollen verursacht wird, ist ein
wichtiger Aspekt Rechter Handlung. Das ist weder ein Vermeiden der
Leidenschaft noch ein Schwelgen in ihr. Zwischen diesen beiden suchen wir den
mittleren Weg.

\end{dhpRefl}

%% == 44 ==

\begin{dhpVerse}{143}
\label{dhp-143}
Ein gut trainiertes Pferd gibt keinen Anlass\\ 
zum Gebrauch der Peitsche.\\ 
Selten sind jene Wesen,\\ 
die aufgrund ihrer Genügsamkeit und Disziplin\\ 
keinen Anlass zu Tadel geben. 
\end{dhpVerse}

\begin{dhpRefl}

Ist es möglich allzu genügsam zu sein? Es ist möglich auf die falsche Art
genügsam zu sein; zum Beispiel, wenn das Verzichten auf zügelloses Verlangen
der Manipulation dient. Genügsamkeit, Bescheidenheit und Disziplin: Begriffe
wie diese können uns Unbehagen bereiten. Doch zweifelsohne sind zeitlose
Prinzipien in ihnen verborgen. Wenn wir uns durch wahre Genügsamkeit
auszeichnen, dann suchen wir in allem nach dem `rechten Maß'. Wir suchen den
Unterschied dazwischen, uns einerseits mit einem `Gut genug' zufrieden zu
geben und andererseits zu zaghaft zu sein, um uns hervorzutun. Mit der Rechten
Disziplin konzentrieren wir uns auf die unmittelbar vor uns liegende Aufgabe,
ohne unser Feingefühl zu vernachlässigen. Jemand, der in der Rechten Disziplin
Geschick entwickelt hat, kann nein sagen, wenn es notwendig ist -- nicht aus
Verurteilung oder bloßer Vorliebe heraus, sondern als jemand, der
\mbox{verantwortungsvoll handelt}.

\end{dhpRefl}

%% == 45 ==

\begin{dhpVerse}{49}
\label{dhp-49}
Wie eine Biene Nektar sammelt\\ 
ohne der Farbe oder dem Duft der Blume\\ 
zu schaden oder diese zu stören,\\ 
so bewegen sich die Weisen durch die Welt. 
\end{dhpVerse}

\begin{dhpRefl}

Die Welt ist unser Dorf. Können wir uns durch die Welt bewegen, ohne das zu
stören, was schön an ihr ist? Weise und Entsagende leben von Almosen, die sie
im Dorf sammeln, die meisten Menschen müssen jedoch Geld verwenden. Gegen Geld
ist an sich nichts einzuwenden. Es ist ein Symbol für die Energie, durch die
wir in der Welt miteinander in Beziehung treten. Die Leitlinien des Buddha für
die Kultivierung des Rechten Lebenserwerbs geben harmlose und umsichtige Wege
im Umgang mit Geld vor. Wenn wir uns vergessen, fixieren wir uns zu sehr
darauf, wie gut es wäre, wenn wir bekämen, was wir haben wollen, und achten
nicht darauf, wie wir uns verhalten, um es zu bekommen. Sorgfalt ist
notwendig, sowohl bei dem, was wir tun, als auch bei der Art und Weise, wie
wir es tun.

\end{dhpRefl}

%% == 46 ==

\begin{dhpVerse}{217}
\label{dhp-217}
Natürlich geliebt sind jene,\\ 
die mit Rechtem Handeln leben,\\ 
die den Weg gefunden und mit Einsicht\\ 
sich in der Wahrheit gefestigt haben. 
\end{dhpVerse}

\begin{dhpRefl}

Wenn unsere körperlichen Handlungen und unsere Sprache Ausdruck innerer
Gelassenheit sind, wird der Bereich der Beziehungen befriedigender für uns.
Wir werden nicht durch die Verfolgung unseres Wunsches geliebt zu werden oder
durch irgendeine andere Befriedigung von Begierden nachhaltig glücklich;
vielmehr geschieht dies durch das Erkennen und Wertschätzen der wahren Gründe
für Zufriedenheit. Sogar heilsame Wünsche können zu Unzufriedenheit führen,
wenn wir uns zu sehr an sie klammern. Das Festhalten ist das Problem. Wenn wir
unsere Gewohnheit des Festhaltens in diesem Moment und an dieser Stelle klar
sehen, erkennen wir, dass wir ihr nicht folgen müssen. So finden wir den
Weg~hinaus.

\end{dhpRefl}

%% == 47 ==

\begin{dhpVerse}{20}
\label{dhp-20}
Nur ein wenig über den Dhamma wissend,\\ 
dafür aber mit ganzem Herzen darauf eingestimmt,\\ 
die Leidenschaften von Gier, Hass und Verblendung umwandelnd,\\ 
alle Anhaftungen an Gegenwärtiges und Zukünftiges loslassend,\\ 
wird man wahrlich für sich selbst den Nutzen\\ 
des Wegbeschreitens erfahren. 
\end{dhpVerse}

\begin{dhpRefl}

Es ist nicht so schlimm, wenn wir nicht alles wissen, was es über den Dhamma
zu wissen gibt. Was zählt ist, wie wir die Lehren leben. Leben wir von ganzem
Herzen gemäß dem, was wir gelernt haben? Das ist eine viel nützlichere
Fragestellung als, `Wann kann ich wieder an einem Retreat teilnehmen?’ oder
`Wenn ich mich nur aufraffen könnte, mehr zu studieren!’ Solche Gedanken sind
in Ordnung, wenn sie uns helfen, Anhaftungen loszulassen, aber nicht, wenn sie
Ausdruck unserer Sucht sind immer mehr zu wollen oder zu werden. Ist eine
subtile Verlagerung der Aufmerksamkeit vonnöten, um im gegenwärtigen Moment
achtsamer zu sein, dann kann sogar das eifrige Streben nach tiefen Einsichten
unseren Fortschritt behindern. Wenn wir nur einen Augenblick lang diese
Wahrheit erkennen, erfahren wir für uns selbst den Nutzen, den das
Beschreiten des Weges für uns bereithält.

\end{dhpRefl}

%% == 48 ==

\begin{dhpVerse}{341}
\label{dhp-341}
Wesen erfahren auf natürliche Weise Freude;\\ 
aber wenn Freude durch Verlangen verunreinigt ist,\\ 
erzeugt Anhaften Frustration und mühsames Leiden ist die Folge. 
\end{dhpVerse}

\begin{dhpRefl}

Mit Rechter Achtsamkeit im richtigen Moment erkennen wir, warum und wie die
Anhaftung an Freude und Kummer Leiden aufrechterhält. Wir beginnen uns dafür
zu interessieren, wie Freude erfahren werden kann, ohne dass ich sie zu
`meiner' Freude mache oder ein `Ich' kreiere, das sich gut unterhält. Weise
Reflexion zeigt uns, dass Glück nicht gemindert, sondern sogar gesteigert
wird, wenn das Erleben nicht mit Anhaftung verbunden ist. Wenn wir in der
richtigen Weise achtsam sind, unterstützt Intelligenz die Einsicht. Wenn wir
begreifen, wie das Anhaften an den Genuss die Schönheit verdirbt, dann entsteht eine
neue Leichtigkeit des~Seins.

\end{dhpRefl}

%% == 49 ==

\begin{dhpVerse}{188-189}
\label{dhp-188}\label{dhp-189}
An viele Orte flüchten sich Wesen aus Angst:\\ 
in die Berge, Wälder, Parkanlagen und Gärten;\\ 
ebenso an heilige Plätze.\\ 
Aber keiner dieser Orte bietet wahre Zuflucht,\\ 
keiner von ihnen kann uns von Angst befreien. 
\end{dhpVerse}

\begin{dhpRefl}

Es ist schwierig Angst zu empfinden, ohne zu meinen, dass etwas nicht in
Ordnung sei. In dem Bestreben dem Schmerz der Angst zu entkommen, reagieren
wir rasch damit, dass wir uns selbst und andere verurteilen. Das funktioniert
aber nicht -- genauso wenig wie die Flucht in die Wildnis. Sogar heilige Orte
werden unsere Erwartungen enttäuschen, wenn unsere Motivation der Wunsch zu
entkommen ist. Wenn wir aber unsere Zuflucht im Dhamma finden, kann das ein
Interesse hervorrufen, Angst zu verstehen und aus der Angst zu lernen. Können
wir die Empfindung von Angst da sein lassen ohne ängstlich zu `werden'? Angst
bleibt Angst, aber sie wird von einer erweiterten, weniger verkrampften und
bedrohten Achtsamkeit wahrgenommen. Wir können sogar anfangen zu sehen, dass
selbst Angst `einfach so' ist. Eine wertfreie, ganzheitliche Anerkennung des
Wesens der Angst im Hier und Jetzt kann unseren Schmerz in Freiheit
verwandeln. Die Bereitschaft uns selbst dort zu begegnen, wo wir uns befinden,
ist der Weg.

\end{dhpRefl}

%% == 50 ==

\begin{dhpVerse}{193}
\label{dhp-193}
Es ist schwierig ein Wesen mit großer Weisheit zu finden;\\ 
selten sind die Orte, in denen sie geboren werden.\\ 
Jene, die sie umgeben, wenn sie erscheinen,\\ 
haben in der Tat großes Glück. 
\end{dhpVerse}

\begin{dhpRefl}

Mit Weisen Zeit zu verbringen, ist sicherlich ein Segen, aber es ist nicht
immer einfach. Ihre Offenheit kann unsere Engstirnigkeit auf schmerzliche
Weise hervorheben. Auf der anderen Seite sind Weise auch mitfühlend, was es
einfacher macht, die `Pille zu schlucken'. Letzten Endes sind wir dankbar.
Nach seiner Erleuchtung empfand der Buddha soviel Dankbarkeit gegenüber seinen
früheren Lehrern, dass er als allererstes begann, sie aufzusuchen, um
weiterzugeben, was er herausgefunden hatte. Auch wir können voller Dankbarkeit
an all jene denken, die uns geholfen haben, Schritte auf diesem Weg zu gehen.
In Dankbarkeit zu verweilen ist Balsam für das Herz; es vereint uns mit den
Weisen.

\end{dhpRefl}

%% == 51 ==

\begin{dhpVerse}{179}
\label{dhp-179}
Des Buddhas Vollkommenheit ist allumfassend;\\ 
es gibt keine Aufgabe mehr zu erledigen.\\ 
Kein Maß gibt es für seine Weisheit;\\ 
keine Grenzen sind auffindbar.\\ 
Auf welche Art und Weise könnte er\\ 
von der Wahrheit abgelenkt sein? 
\end{dhpVerse}

\begin{dhpRefl}

Es gibt eine wahre Realität, die erkannt werden kann; der Buddha kannte sie.
Es ist möglich jenseits aller begrenzenden Gefühle zu erwachen, die zu
Enttäuschung und Bedauern führen. Angemessenes Vertrauen in die Erleuchtung
des Buddha schafft einen inneren Bezugspunkt. Wenn wir die Orientierung im
Leben verloren haben, finden wir sie wieder, indem wir uns auf dieses
Vertrauen einlassen. Vertrauen ist eine innere Struktur, wie die Segel einer
Yacht; richtig positioniert, fangen sie den Wind ein, und so gelangt das Schiff
an seinen Bestimmungsort.

\end{dhpRefl}

%% == 52 ==

\begin{dhpVerse}{153-154}
\label{dhp-153}\label{dhp-154}
Für viele Leben bin ich, suchend nach dem Hausbauer,\\ 
der mein Leiden verursachte, erfolglos umhergewandert.\\ 
Aber jetzt bist du erkannt und sollst nichts mehr bauen.\\ 
Die Querbalken sind entfernt und die Firstbalken zerbrochen.\\ 
Alles Verlangen ist beendet; mein Herz ist eins\\ 
mit dem Unbedingten. 
\end{dhpVerse}

\begin{dhpRefl}

Die endgültige und vollständige Erkenntnis des Buddha bestand darin zu
realisieren, dass er an etwas geglaubt hatte, was unwirklich war. Er wurde von
dem getäuscht, was er als den `Hausbauer' bezeichnete. Die Häuser sind die
Strukturen des Geistes; das `Mich' und das `Mein', die wir so ernst nehmen.
`Ich bin es, der sich sehnt.‘ `Ich bin es, der sich enttäuscht fühlt.’ `Das
sind meine Laune, mein Körper und mein Geist.’ Er sah klar und deutlich, wie
diese Eindrücke durch die Gewohnheiten des Verlangens geschaffen wurden. Als
dies mit tiefer Einsicht durchdrungen worden war, war der Prozess verstanden;
die Hauptstütze des Hauses, der Firstbalken, war zerbrochen und das Leiden
beendet. Das Umherirren in der Hoffnung, eine Lösung für seine Unzufriedenheit
zu finden, war beendet. Von da an verweilte er behaglich in dem Zustand der
allem Entstehen und Vergehen vorausgeht -- dem Unbedingten, der
unvergänglichen Wirklichkeit.

\end{dhpRefl}
